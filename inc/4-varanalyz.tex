\section{Построение графика заданной функции}

Теперь нужно сделать так, чтобы вместо пустого графика строился график какой-то функции. Далее будет использован немного измененный код из примеров использования библиотеки QCustomPlot, а именно код https://www.qcustomplot.com/index.php/tutorials/basicplotting. Приблизительно то, что описано ниже, можно посмотреть в видео https://www.youtube.com/watch?v=xWGEvlDWokQ (на английском языке).

В визуальном конструкторе выделим widget с графиком, нажав на него в конструкторе или на "widget" в списке объектов. Справа внизу в свойствах объекта меняем objectName на "customPlot". Можно использовать любое другое имя, однако в используемых примерах кода именно "customPlot", в случае смены имени придется заменить "customPlot" на ваше имя. Где находится настройка значения параметар objectName, показано на скриншоте.
\addimghere{qtc41}{1}{Смена objectName}{qtc41}

В дереве проекта выбрать и открыть для редактирования путем двойного нажатия файл *.cpp, в данном случае это graqa.cpp. В него в конец дописать функцию рисования графика, пока что пустую:
\begin{lstlisting}[language=c++]
void MainWindow::makePlot()
{

}
\end{lstlisting}
\addimghere{qtc42}{1}{Добавление функции рисования графика}{qtc42}

Теперь нажмите правой кнопкой мыши в любое место первой строки этого кода, выберите: "Рефакторинг" - "Добавить объявления private slots", как показано на скриншоте.
\addimghere{qtc43}{1}{Меню рефакторинга}{qtc43}

В файле graqa.h появился код:
\begin{lstlisting}[language=c++]
private slots:
    void makePlot();
\end{lstlisting}
\addimghere{qtc44}{1}{Автоматически сгенерированный код}{qtc44}

В graqa.cpp в код рисования окна допишем вызов функции рисования графика, как показано на скриншоте. При написании кода будет в Qt Creator будет срабатывать автоматическое дополнение кода, чтобы согласиться с предлагаемым вариантом автодополнения, можно нажать Enter.
\begin{lstlisting}[language=c++]
MainWindow::makePlot();
\end{lstlisting}
\addimghere{qtc45}{1}{Вставка вызова функции рисования графика}{qtc45}

Теперь в тело функции рисования графика makePlot вставим код из документации к библиотеке QCustomPlot (https://www.qcustomplot.com/index.php/tutorials/basicplotting), как показано на скриншоте. Код примера приведен ниже после скриншота, но комментарии переведены на русский язык.
\addimghere{qtc46}{1}{Вставка типового кода рисования графика}{qtc46}

\begin{lstlisting}[language=c++]
// заполненим переменные произвольными значениями:
QVector<double> x(101), y(101); // initialize with entries 0..100
for (int i=0; i<101; ++i)
{
  x[i] = i/50.0 - 1; // x идет от -1 до 1
  y[i] = x[i]*x[i]; // будем строить график квадратичной функции
}
// создаем график и привязываем данные к нему
customPlot->addGraph();
customPlot->graph(0)->setData(x, y);
// установим наименования (подписи) осей
customPlot->xAxis->setLabel("x");
customPlot->yAxis->setLabel("y");
// установим диапазон значений по обеим осям
customPlot->xAxis->setRange(-1, 1);
customPlot->yAxis->setRange(0, 1);
// перерисуем график с новыми параметрами
customPlot->replot();
\end{lstlisting}

На скриншоте ниже отмечена структура ui в типовом проекте Qt Creator. Там создается главное окно программы (MainWindow). Нужно в коде функции makePlot сделать так, чтобы customPlot относилось именно к ui, т.к. график будет рисоваться внутри MainWindow. Для этого заменяем "customPlot->" на "ui->customPlot->".
\addimghere{qtc47}{1}{ui}{qtc47}

Теперь выполняем сборку и запуск программы. Откроется окно программы с графиком заданной функции.
\addimghere{qtc48}{1}{Сборка и запуск программы}{qtc48}

Получилась вполне рабочая программа. Можно менять x[i] и y[i] в коде, чтобы рисовать график другой функции. При изменении размеров окна программы размер графика также изменяется.

Выполните фиксирование изменений в Git, как уже было рассмотрено ранее ("Инструменты" - "Git" - "Локальное хранилище" - "Фиксировать (commit)" или git commit -a в консоли).

Теперь зададим другую математическую функцию для построения графика вместо той, что была в примере. Нам понадобятся математические функции и контсанты, например, число Пи. Для этого подключим заголовочный файл qmath.h в главном заголовочном файле нашей программы:
\begin{lstlisting}[language=c++]
#include <qmath.h>
\end{lstlisting}
\addimghere{qtc49}{1}{Подключение qmath}{qtc49}

В qmath.h задано число Пи константой M\_PI:
\begin{lstlisting}[language=c++]
#define M_PI (3.14159265358979323846)
\end{lstlisting}

\clearpage
