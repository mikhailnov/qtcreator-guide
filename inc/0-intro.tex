\anonsection{Введение}

В данном руководстве рассматривается работа с программой Qt Creator — интегрированной средой средой разработки (IDE, Integrated Development Environment) программного кода, преимущественно с использованием программного фреймфорка Qt.

Отличительной особенностью фреймворка Qt и IDE Qt Creator в частности являются кроссплатформенность, открытый исходный код и свободная лицензия.

Qt распространяется под лицензией LGPGL (Light GNU Public License), что позволяет не только свободно модифицировать и распространять Qt, но и в отличие от лицензии GPL разрешает не открывать исходный код производных продуктов при условии не внесения изменений в код Qt, иначе требуется либо открыть код модифицированной версии Qt, либо купить коммерческую лицензию у компании-разработчика Qt.

Фреймворк Qt позволяет создавать как графические, так и неграфические консольные приложения, однако наибольшее распространение имеет для создания именно графических приложений, позволяя из единой кодовой базы создавать кроссплатформенное приложение, то есть такую программу, которая может быть скомпилирована и запущена на разных операционных системах, например, GNU/Linux, Android, Windows, Mac OS, FreeBSD, DragonFlyBSD, NetBSD, Haiku и др.

В данном руководстве рассмотрен фреймворк Qt версии 5, однако большая часть информации должна остаться актуальной для его будущих версий.

Снятие скриншотов и сборка приведенного программного кода выполнялись в ОС GNU/Linux Ubuntu 19.04 с Qt Creator 4.8.1, Qt 5.12.2 и компиляторами GCC 8.3.0 и LLVM Clang 8.0.0. Приведенный код должен остаться рабочим в будущих версиях указанных компиляторов и на других операционных системах, в т.ч. Windows и Mac OS. Тестирование работоспособности кода выполнено в GNU/Linux и DragonFlyBSD.

\clearpage
