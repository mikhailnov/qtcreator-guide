\section{Создание проекта в Qt Creator}

Интегрированная среда разработки Qt Creator предназначена для проектирования графического интерфейса программы на Qt и для разработки кода программ на Qt.

Следует понимать, что сам Qt Creator не занимается сборкой (то есть компилированием) программного кода, он лишь с нужными параметрами вызывает компилятор и/или систему сборки (qmake, cmake и др.). Можно проделать эти операции через командную строку вручную.

Открываем программу Qt Creator.
\addimghere{qtc1}{1}{Главное окно Qt Creator при запуске}{qtc1}

Нажимаем на кнопку "Новый проект".
\addimghere{qtc2}{1}{Главное окно Qt Creator при запуске}{qtc2}

Откроется окно с выбором типа создаваемого проекта. Выбираем: "Приложение - "Приложение Qt Widgets". Нажимаем "Выбрать".
\addimghere{qtc3}{1}{Выбор типа проекта Qt}{qtc3}

В следующем окне указываем название проекта латинскими буквами без пробелов и папку, в которой будет создана отдельная подпапка с именем проекта, внутри которой будет размещен сам проект. Данный проект-пример будет назван "graqa" (читать как "Грака"). Нажимаем "Далее".
\addimghere{qtc4}{1}{Название и размещение проекта}{qtc4}

Далее нужно указать параметры сборки проекта. Для раскрытия показанного на скриншоте списка параметров нажать на "Подробнее". Если навести мышь слева от кнопки "Подробнее", появится кнопка "Управление...".
\addimghere{qtc5}{1}{Выбор комплекта}{qtc5}

Можно оставить параметры по умолчанию, однако рекомендуется нажать на кнопку "Управление..." и проверить параметры конфигурации. На ОС Linux в стандартной конфигурации "Desktop" обычно указан компилятор GCC. Принципиальной разницы между GCC и Clang в данном случае не будет, однако у Clang обычно более понятные тексты ошибок, поэтому в данном случае я установлю Clang в настройках профиля "Desktop". Другие правки не вношу. После внесения изменений или при их отсутствии нажать "ОК" для возврата к предыдущему окну. Нажимать "Далее".
\addimghere{qtc6}{1}{Редактирование профиля Desktop}{qtc6}

Следующий шаг — "Информация о классе". На скриншоте ниже показаны параметры по умолчанию.
\addimghere{qtc7}{1}{Информация о классе — умолчания}{qtc7}

Для удобства переименуем "mainwindow.cpp", "mainwindow.ui" и "mainwindow.h" так, чтобы вместо "mainwindow" было название проекта, в данном случае "graqa". Это делать не обязательно, но позволит не запутаться.
\addimghere{qtc8}{1}{Информация о классе — переименования}{qtc8}

Последний шаг в настройке проекта Qt Creator — проверка создаваемых файлов. В поле "Добавить под контроль версий", если установлен Git, рекомендуется указать его, чтобы можно было фикировать изменения файлов и просматривать их историю. При необходимости нажать "Назад", внести правки в настрйоки и нажать "Далее". Нажать "Завершить".
\addimghere{qtc9}{1}{Итог}{qtc9}

Откроется созданный по шаблону проект. Дальнейшая работа с ним будет рассмотрена в следующих разделах.
\addimghere{qtc10}{1}{Созданный из шаблона проект в Qt Creator}{qtc10}

Если при создании проекта не была выбрана система контроля версий, то нужно создать хранилище git (git init). Для этого выберем: "Инструменты" - "Git" - "создать хранилище", откроется окно с выбором пути для сохранения хранилища, путем по умолчанию будет папка проекта, ее и оставим, нажав "ОК" или "Выбрать".
\addimghere{qtc11}{1}{git init}{qtc11}

Если используется git, рекомендуется зафиксировать изначальное состояние файлов проекта, созданных из шаблона при создании проекта. Для этого в меню выбрать: "Инструменты" - "Git" - "Локальное хранилище" - "Фиксировать (commit)". Откроется окно фиксации git. В git у каждого коммита, т.е. у каждой зафикированной порции изменений файлов, обязательно должны быть имя и email автора. Если у пользователя глобально не настроен git, то поля "Автор" и "E-mail" наверянка придется заполнить вручную. Обязательно ввести "Описание" (commit message), отражающее суть правок. Для коммита с шаблоном напишем, например, так: "Init from template" ("Создано из шаблона"), писать текст коммитов можно на любом языке, в т.ч. русском. Нажать "Фиксировать N из M файлов".
\addimghere{qtc12}{1}{Окно фиксации Git}{qtc12}

Свернуть появившийся внизу лог действий можно нажатием на кнопку справа, как показано на скриншоте ниже.
\addimghere{qtc13}{1}{Сворачивание окна с логом}{qtc13}

Однако вместо описанных выше действий с Git в графическом интерфейсе Qt Creator лучше научиться работать с git в консоли, т.к. это изначально консольное приложение с мощным консольным интерфейсом, а Qt Creator лишь обладает графической оберткой над ним, в которой реализован очень далеко не весь функционал. Если вы хотите проделать все действия с git в консоли, то откройте консоль и выполните описанные ниже действия (команды):

\begin{lstlisting}[language=bash]
# настроить автора в git
git config --global user.name "Vasya Pupkin" vasya@yandex.ru
# создать git-репозиторий
git init
# добавить все файлы в очередь коммита (stage)
git add .
# сделать коммит
git commit -a
\end{lstlisting}

\clearpage
