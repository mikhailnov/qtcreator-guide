\section{Подготовка программного окружения для разработки. Установка Qt Creator}

Программное окружение для разработки приложений с использованием фреймворка Qt состоит из:
\begin{itemize}
  \item поддерживаемая Qt операционная система;
  \item установленные компоненты Qt для разработчиков;
  \item компилятор C++;
  \item система контроля версионирования файлов Git или другая (не обязательна, но настоятельно рекомендуется);
\end{itemize}

\subsubsection{Установка Qt на Windows, Mac OS}
\begin{itemize}
  \item Открыть веб-страницу https://www.qt.io/download в браузере;
  \item Выбрать "Go open source", как показано на рисунке 1;
  \item Загрузить онлайн или оффлайн установщик, при установке не забыть отметить для установки набор компилятором MinGW
  \item Выполнить установку и проверить установленную программу Qt Creator на запускаемость;
  \item Рекомендуется установить Git (https://git-scm.com/book/ru/v1/Введение-Установка-Git);
\end{itemize}
\addimghere{qt_win_1}{1}{Выбор модели распространения Qt}{qt_win_1}

Для построения графиков будет использоваться библиотека QCustomPlot. На Windows придется скачать с сайта https://www.qcustomplot.com/ в разделе "Download" архив QCustomPlot.tar.gz, из которого файлы ./qcustomplot.cpp и ./qcustomplot.h нужно будет в дальнейшем вручную добавить в проект Qt Creator. В случае Linux/BSD библиотека QCustomPlot есть в репозитории большинства дистрибутивов, достаточно поставить ее оттуда, приведенные ниже списки пакетов для Linux/BSD уже содержат ее, однако можно включить скачанные с сайта файлы в проект. При установке из репозитория ничего скачивать и добавлять в проект не нужно.

\subsubsection{Установка Qt на GNU/Linux и BSD-системы}
На Linux можно так же скачать и поставить полный набор для разработки на Qt с сайта Qt, однако гораздо удобнее и быстрее поставить все необходимое из репозитория используемого дистрибутива.

Установка на Ubuntu/Mint/Debian:
\begin{lstlisting}[language=bash]
sudo apt install --install-recommends build-essential clang qtcreator git libqcustomplot-dev
\end{lstlisting}

Установка на ROSA:
\begin{lstlisting}[language=bash]
sudo urpmi basesystem-build clang qt-creator git lib64qcustomplot-qt5-devel
\end{lstlisting}

Установка на FreeBSD/DragonFlyBSD:
\begin{lstlisting}[language=bash]
sudo pkg install qtcreator git
\end{lstlisting}

\clearpage
